\section{Vectors and Matrices}
    
    \subsection{Vectors}
        Vector Addition
        \begin{equation}
            \boldsymbol{v} + \boldsymbol{w} = 
            \begin{bmatrix}
                v_1 \\
                v_2
            \end{bmatrix}
            +
            \begin{bmatrix}
                w_1 \\
                w_2
            \end{bmatrix}
            =
            \begin{bmatrix}
                v_1 + w_1 \\
                v_2 + w_2
            \end{bmatrix}
        \end{equation}
        Scalar Multiplcation
        \begin{equation}
            2\boldsymbol{v} = 
            \begin{bmatrix}
                2v_1 \\
                2v_2
            \end{bmatrix}
        \end{equation}

    \subsection{Matrices}

        \subsubsection{Matrix Multiplcation}
            A linear combination of three column vectors is seen as, 
            \begin{equation}
                x_1
                \begin{bmatrix}
                    1 \\
                    -1 \\
                    0
                \end{bmatrix}
                +
                x_2
                \begin{bmatrix}
                    0 \\
                    1 \\
                    -1
                \end{bmatrix}
                +
                x_3
                \begin{bmatrix}
                    0 \\
                    0 \\
                    1
                \end{bmatrix}
                =
                \begin{bmatrix}
                    x_1 \\
                    -x_1 \\
                    0
                \end{bmatrix}
                +
                \begin{bmatrix}
                    0 \\
                    x_2 \\
                    -x_2
                \end{bmatrix}
                +
                \begin{bmatrix}
                    0 \\
                    0 \\
                    x_3
                \end{bmatrix}
                =
                \begin{bmatrix}
                    x_1 \\
                    -x_1 + x_2\\
                    -x_2 + x_3
                \end{bmatrix}
            \end{equation}
            This can be expressed in the form of \(\boldsymbol{Ax}=\boldsymbol{b}\), by expressing the vectors in matrix form,
            \begin{equation}
                \boldsymbol{Ax}=\boldsymbol{b} \quad \rightarrow \quad
                \begin{bmatrix}
                    1  & 0  & 0 \\
                    -1 & 1  & 0 \\
                    0  & -1 & 1
                \end{bmatrix}
                \begin{bmatrix}
                    x_1 \\
                    x_2 \\
                    x_3
                \end{bmatrix}
                =
                \begin{bmatrix}
                    x_1 \\
                    -x_1 + x_2\\
                    -x_2 + x_3
                \end{bmatrix}
            \end{equation}
            This shows matrix multiplication by columns, compared to the more traditional way of thinking which is multiplication a row at a time using the dot product,
            \begin{equation}
                \begin{bmatrix}
                    1  & 0  & 0 \\
                    -1 & 1  & 0 \\
                    0  & -1 & 1
                \end{bmatrix}
                \begin{bmatrix}
                    x_1 \\
                    x_2 \\
                    x_3
                \end{bmatrix}
                =
                \begin{bmatrix}
                    \boldsymbol{a}_1 \\
                    \boldsymbol{a}_2 \\
                    \boldsymbol{a}_3
                \end{bmatrix}
                \boldsymbol{x}
                =
                \begin{bmatrix}
                    \boldsymbol{a}_1 . \boldsymbol{x} \\
                    \boldsymbol{a}_2 . \boldsymbol{x} \\
                    \boldsymbol{a}_3 . \boldsymbol{x}
                \end{bmatrix}
            \end{equation}

        \subsubsection{Inverse Matricies Basics}
            Matrix multiplication is useful in the expression \(\boldsymbol{Ax}=\boldsymbol{b}\), when solving for 
            \(\boldsymbol{b}\). 
            Inverses matrices are needed when solving for \(\boldsymbol{x}\), in the form of 
            \(\boldsymbol{x}=\boldsymbol{A}^{-1} \boldsymbol{b}\), where \(A^{-1}\) is the inverse of \(\boldsymbol{A}\).
            
            \par \hfill \break
            Using the matrix in the above example, the inverse can be seen as,
            \begin{equation}
                \boldsymbol{Ax}=\boldsymbol{b} 
                \quad \rightarrow \quad
                \begin{bmatrix}
                    x_1 \\
                    -x_1 + x_2\\
                    -x_2 + x_3
                \end{bmatrix}
                =
                \begin{bmatrix}
                    b_1 \\
                    b_2 \\
                    b_3
                \end{bmatrix}
                \qquad \qquad
                \boldsymbol{x}=\boldsymbol{A}^{-1} \boldsymbol{b}
                \quad \rightarrow \quad
                \begin{bmatrix}
                    x_1 \\
                    x_2 \\
                    x_3
                \end{bmatrix}
                =
                \begin{bmatrix}
                    b_1 \\
                    b_1 + b_2 \\
                    b_1 + b_2 + b_3
                \end{bmatrix}
            \end{equation}
        
        \subsubsection{Cyclic Differences}
            Considering the examples above, the matrix \(\boldsymbol{A}\) is a difference matrix, as the \(\boldsymbol{b}\) solutions are differences of one another.
            \begin{equation}
                \begin{bmatrix}
                    1  & 0  & 0 \\
                    -1 & 1  & 0 \\
                    0  & -1 & 1
                \end{bmatrix}
                \begin{bmatrix}
                    x_1 \\
                    x_2 \\
                    x_3
                \end{bmatrix}
                =
                \begin{bmatrix}
                    x_1 \\
                    x_2 - x_1\\
                    x_3 - x_2
                \end{bmatrix}
            \end{equation}
            In order to turn this matrix into a cyclic matrix the \(A_{13}\) is changed to \(-1\) which means that \(b_1 = x_1-x_3\), hence the matrix will now be,
            \begin{equation}
                \boldsymbol{C} =
                \begin{bmatrix}
                    1  & 0  & -1 \\
                    -1 & 1  & 0 \\
                    0  & -1 & 1
                \end{bmatrix}
            \end{equation}
            This is not a triangular matrix, and the three equations have either an infinite number of solutions or, the usual case, no solutions at all.
        
        \subsubsection{Independance and Dependance}
        TODO